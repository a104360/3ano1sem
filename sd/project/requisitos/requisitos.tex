\documentclass{article}[12pt]
\usepackage{enumitem}

\title{\textbf{\Large{Requisitos funcionais de SD}}}

\begin{document}
	\maketitle
	\pagebreak
	\textbf{Autenticar e registar utilizador}
	\begin{enumerate}
		\item Registar utilizador 
		\item Autenticar utilizador
		\item A interação com o serviço implica\\
			Estabelecer conexão\\
			Autenticar o utilizador
		
	\end{enumerate}
	
	\textbf{Operações leitura/escrita simples}
	\begin{enumerate}
		\item \textbf{Escrita} - void put(String key, byte[] value);\\
				Operações de escrita são baseadas numa função, chave-valor, onde a chave é o "ficheiro" para onde se vai escrever, e o valor é o texto que se vai escrever.\\
				Caso não haja entrada (c.f. ficheiro não existe), é criada uma nova entrada no servidor.
		\item \textbf{Leitura} - byte[] get(String key);\\
				Operações de escrita são baseadas na função get que tem como argumento apenas a entrada do "ficheiro" que será lido.\\
				Deve retornar null caso não exista a entrada.
	\end{enumerate}
	
	\textbf{Operações de leitura/escrita composta}
	\textbf{Escrita} - \texttt{void multiPut(Map<String, byte[]> pairs);}\\
	\textbf{Leitura} - \texttt{Map<String,byte[]> multiGet(Set<String> keys);}\\
	
	
	\textbf{Utilizadores concurrentes}
	Deve ser configurado um máximo\\
	Se for atingido, os utilizadores devem ser colocados numa fila de espera\\

	
	
	\textbf{\Large{Funcionalidade Avançada}}
	
	\textbf{Clientes multiThread}\\
	Clientes devem poder ter várias threads a responder aos seus pedidos

	\textbf{Leitura condicional}\\
	\textbf{byte[] getWhen(String key, String keyCond, byte[] valueCond);}\\
	Deve devolver o valor de key, sse o valor de keyCond seja valueCond, deve bloquear a operação até acontecer.\\


	\textbf{Programas a implementar}\\
	\textbf{Servidor}\\
	\textbf{Biblioteca do cliente}\\
	\textbf{Interface}\\
	

\end{document}
